\documentclass[12pt]{article}
\usepackage{graphicx}
\usepackage{color}
\usepackage{fixltx2e}
\usepackage{datatool}
\usepackage{float}
%\usepackage[showframe=true]{geometry}
\addtolength{\textwidth}{1in}
\addtolength{\textheight}{1in}
\addtolength{\evensidemargin}{0.5in}
\addtolength{\oddsidemargin}{-0.5in}
\addtolength{\topmargin}{-0.5in}
\setlength{\parindent}{0cm}

\begin{document}






\section{Hardware}

The hardware exists in two versions: a minaturised version intended
for flight and a development version intended for firmware
developement and debugging etc. The two verisons share the same
circuit diagram, and hence firmware should be generally interoperable
between the two.

However to allow detection of the hardware version PA02 is shorted to
ground on the development version while is will be pulled high
internally on the flight version.

\subsection{Dimensions}

The flight hardware has a total size of 50 x 12 mm.

Components are positioned on a 1/16 (0.0625) mm grid, with major
components on a 1/4 mm grid.

\subsection{Routing}

The board is routed with 6 mil traces for data and 6 / 16 mil traces
for power. Where nessesary the traces  All via are 0.3mm (12mil) drill.

\section{Components}

\subsection{Development}

The

\section{Design}

\subsection{HF Clock}

This is synthesised in the SAM D20E and used to drive the XIN input of
the SI4060. The design frequency for the XIN input is 25 - 32MHz
(30MHz by default) but UPU reports success with 16.369MHz. The maximum
output clock frequency from the SAM D20E is 48MHz so this would be a
useful design limit.

No input capacitance value is given for the XIN pin but 2pF typical is
given for out pins and this is probably about the same.

The maximum rise/fall time for 48MHz is 10ns, around 5ns would be
good, giving a knee = 100MHz.

Path inductance is maybe 10nH total (1+1+1 x 2 for pad/gnd-pad/via, 4
for trace)

Maximum output current from the SAM D20E is 1mA (with DRVSTR=0, 1.8V),
so output impedance = 1800Ω

Worst case Q
\begin{equation}
  Q = {\sqrt{10nH/2pF} \over 100} = 0.22 (good)
\end{equation}

Time constant
\begin{equation}
  tc = RC = 1800*2pF = 3.6ns (good)
\end{equation}

In any case the SI4060 tune API allows ramping up the internal
capacitance to 11pF, which will drop Q at the expense of the time
constant.

\subsection {Firmware}

Check flight WG2 on habhub for gps string formatting gone wrong

Get so much test data.

\end{document}